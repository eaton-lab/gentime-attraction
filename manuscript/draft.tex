% Use only LaTeX2e, calling the article.cls class and 12-point type.

\documentclass[12pt]{article}

% Users of the {thebibliography} environment or BibTeX should use the
% scicite.sty package, downloadable from *Science* at
% http://www.sciencemag.org/authors/preparing-manuscripts-using-latex 
% This package should properly format in-text
% reference calls and reference-list numbers.

\usepackage{scicite}
\usepackage{times}
\usepackage{siunitx}
\usepackage{url}
\urlstyle{same}

% allows inserting figures.
\usepackage[pdftex]{graphicx}

% allows multirows in tables.
\usepackage{multirow}
\usepackage{booktabs}

% allows separate/sequential bibliography for main text and supplement.
% \usepackage{filecontents}
% \usepackage{biblatex}


% The preamble here sets up a lot of new/revised commands and
% environments.  It's annoying, but please do *not* try to strip these
% out into a separate .sty file (which could lead to the loss of some
% information when we convert the file to other formats).  Instead, keep
% them in the preamble of your main LaTeX source file.


% The following parameters seem to provide a reasonable page setup.

\topmargin 0.0cm
\oddsidemargin 0.2cm
\textwidth 16cm 
\textheight 21cm
\footskip 1.0cm


%The next command sets up an environment for the abstract to your paper.

\newenvironment{sciabstract}{%
\begin{quote} \bf}
{\end{quote}}



% Include your paper's title here
\title{Generation time branch attraction biases the sequential coalescent}

% Place the author information here.  Please hand-code the contact
% information and notecalls; do *not* use \footnote commands.  Let the
% author contact information appear immediately below the author names
% as shown.  We would also prefer that you don't change the type-size
% settings shown here.

\author
{Patrick F. McKenzie,${^1}$, Sandra L. Hoffberg,${^1}, and Deren A. R. Eaton,$^{1\ast}$\\
% {John Smith,$^{1\ast}$ Jane Doe,$^{1}$ Joe Scientist$^{2}$\\
\\
\normalsize{$^{1}$Department of Ecology, Evolution, and Environmental Biology, Columbia University,}\\
\normalsize{1200 Amsterdam Ave., New York, USA}\\
\\
\normalsize{$^\ast$To whom correspondence should be addressed; E-mail:  de2356@columbia.edu.}
}

% Include the date command, but leave its argument blank.

\date{}



%%%%%%%%%%%%%%%%% END OF PREAMBLE %%%%%%%%%%%%%%%%



\begin{document} 

% Double-space the manuscript.

\baselineskip24pt

% Make the title.

\maketitle 



% Place your abstract within the special {sciabstract} environment.

\begin{sciabstract}
  The multispecies coalescent (MSC) is a model that can predict the 
  expected distribution of unlinked genealogies across a whole genome 
  alignment. For most applications of the MSC edge lengths between 
  divergence events in a species tree (demographic model topology)
  are described in coalescent units, a compound measurement representing 
  both time in generations (g) and the effective population size (Ne), 
  since these two parameters have equal effects on the probability of 
  coalescence. In the case of linked genealogies, however, these parameters
  must be separated. Both can affect the variance of coalescent times, 
  and the variance in similarity between two genealogies, but g has the 
  addition effect in determining the similarity of neighboring genealogies.
  Here we show lineages with long generation times can experience 
  artifactual attraction in concatenation analyses. Moreover, we show that
  gene trees inferred from concatelescence within gene regions also exhibits
  the same effect. This biases both species tree inference and network 
  inference based on gene trees. Our results suggest that for deeper time
  analyses SNP-based approaches that avoid gene tree concatelescence 
  effects are more accurate.

\end{sciabstract}

% In setting up this template for *Science* papers, we've used both
% the \section* command and the \paragraph* command for topical
% divisions.  Which you use will of course depend on the type of paper
% you're writing.  Review Articles tend to have displayed headings, for
% which \section* is more appropriate; Research Articles, when they have
% formal topical divisions at all, tend to signal them with bold text
% that runs into the paragraph, for which \paragraph* is the right
% choice.  Either way, use the asterisk (*) modifier, as shown, to
% suppress numbering.


\newpage

% \subsection*{Main text} 
\noindent 
The multispecies coalescent makes the following assumptions and uses
the following units. Here we show that x,y,z.





\subsection*{Supporting Online Material}
Materials and Methods\\
Figs. S1 to S8\\
Tables S1 to S2\\
References \textit{(33-39)}


\subsection*{Acknowledgments}
This work was supported by an NSF grant to D.E. (DEB-1557059), 
and Columbia University startup funds. Thanks to B. Haller for 
advice on improving SLiM3 code, and to feedback on earlier versions
of this manuscript from T. Price.
% A reproducible science document is available in Supplement XXX that 
% combines code from SLiM3 (Eidos), Python, and R to perform all 
% simulations, parse results, and generate figures and statistical results.


% RUN ONCE TO PRODUCE .BBL FILES
% \bibliographystyle{Science}
% \bibliography{DMIs_Polished_NOabbrev}


\clearpage


% \begin{figure}
%   \centering{
%     \includegraphics[width=0.99\textwidth]{../figures/Fig-1}
%   }
% \end{figure}
\noindent {\bf Figure 1.}
  Linked versus unlinked genealogies and gene trees. Genealogies cannot be 
  observed directly, only inferred from the information contained in sequence
  data. Empirical gene trees often combine multiple true genealogies, 
  especially at deep time scales when recombination has great effect.
\clearpage






\section*{Materials and Methods}

\subsection*{General simulation framework}




% \renewcommand\refname{Materials and Methods References}
% \begin{thebibliography}{10}
%   \makeatletter
%   \addtocounter{\@listctr}{32}
%   \makeatother

% \bibitem{cattani_genetics_2009}
% M.~V. Cattani, D.~C. Presgraves, {\it Genetics\/} {\bf 181}, 1545 (2009).

% \bibitem{cutter_polymorphic_2012}
% A.~D. Cutter, {\it Trends in Ecology \& Evolution\/} {\bf 27}, 209 (2012).

% \bibitem{wang_evolution_2013}
% R.~J. Wang, C.~Ané, B.~A. Payseur, {\it Evolution\/} {\bf 67}, 2905 (2013).

% \bibitem{orr_evolution_2001}
% H.~A. Orr, M.~Turelli, {\it Evolution\/} {\bf 55}, 1085 (2001).

% \bibitem{elzhov_minpacklm_2016}
% T.~V. Elzhov, K.~M. Mullen, A.-N. Spiess, B.~Bolker, minpack.lm: {R} interface
%   to the {Levenberg}-{Marquardt} nonlinear least-squares algorithm found in
%   {MINPACK}, plus support for bounds. {R} package version 1.1-8 (2016).

% \bibitem{mazerolle_AICcmodavg_2019}
% M.~J. Mazerolle, {AICcmodavg}: {Model} selection and multimodel inference based
%   on ({Q}){AIC}(c) (2019).

% \bibitem{webber_similarity_2010}
% W.~Webber, A.~Moffat, J.~Zobel, {\it ACM Transactions on Information Systems\/}
%   {\bf 28}, 20:1 (2010).

% \end{thebibliography}
\clearpage



% FIGURE S1: CARTOON DRAWING OF SIMULATION
% \begin{figure}
%   \centering{
%     \includegraphics[width=0.99\textwidth]{../figures/SFigure_Dem_VS_Orr_cartoon_2}
%   }
% \end{figure}
\noindent {\bf Figure S1.}
A demonstration of the correlation between X and Y in simulated genealogies...

\clearpage


\end{document}




















